\chapter{Постановка задачи}
\label{sec:Chapter1} \index{Chapter1}
С учётом принятой терминологии цель работы и решённые задачи выражаются следующим образом.
\section{Цель работы}
Целью работы является разработка модуля, способного каким-либо образом вычислять заранее обговорённый набор платформо-зависимых констант для интересующих разработчика платформ, а так же имеющего интерфейс, предоставляющий доступ к значению любой из констант на каждой из архитектур.
\par
\section{Решённые задачи}
Для достижения этой цели были решены следующие задачи.
\begin{enumerate}
    \item Найден способ вычисления гостевых констант на хост-платформе (с одновременной поддержкой нескольких платформ).
    \item Разработан интерфейс, предоставляющий компилятору виртуальной машины доступ к множеству вычисленных гостевых констант.
    \item Обеспечена синхронизация между процессом вычисления констант и интерфейсом, использующимся компилятором.
\end{enumerate}
При решении этих задач особенно важным являлся вопрос масштабируемости (касательно как поддержки новых платформ, так и добавления новых/удаления неиспользуемых платформо-зависимых констант) и автоматизации, то есть минимизации ручной работы, требующейся для корректного функционирования этого модуля при модификации виртуальной машины. 
Также произведена оптимизация задержек, вносимых процессом обеспечения вышеупомянутой синхронизации. Подробное описание этих аспектов можно найти в главах \ref{sec:Chapter3} и \ref{sec:Chapter4}, посвящённых деталям и недостаткам реализации.

\newpage
