\chapter{Постановка задачи}
\label{sec:Chapter1} \index{Chapter1}

Примем следующие определения.

\begin{itemize}
    \item
        \textit{Тулчейн} - набор утилит, позволяющий транслировать C++-программы в некоторый бинарный код.
    \item
        \textit{Платформа} - предполагаемое тулчейном окружение, в котором будет исполняться сгенерированный им код. Подразумевается, что каждая платформа в достаточной степени однозначно определяется некоторым тулчейном. В качестве синонима, в данной работе будет также использоваться термин \textit{архитектура}.
    \item
        \textit{Платформо-зависимая константа} - имеющее определённое имя, определённое в терминах C++ константное выражение, численное значение которой каким-либо образом зависит от платформы.
\end{itemize}

Целью данной работы является разработка модуля, способного каким-либо способом вычислять заранее обговорённый набор платформо-зависимых констант для интересующих разработчика платформ, а так же имеющего интерфейс, предоставляющий доступ к значению любой из констант на каждой из архитектур.
\newpage