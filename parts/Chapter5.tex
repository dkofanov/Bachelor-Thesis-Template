\chapter{Заключение}
\label{sec:Chapter5} \index{Chapter5}
В процессе работы был разработан и интегрирован модуль автогенерации платформо-зависимых констант, отличающегося высокой степенью автоматизированности и масштабируемости, а также обладающего легковесным и интуитивно понятным С++-интерфейсом.
Его применение позволяет эффективно разрешить одну из проблем кросс-компиляции, то есть генерации нативного кода для гостевой платформы, в контексте AOT-компилятора виртуальной машины.
Также произведено сравнение этого решения с альтернативным, использующимся для кросс-компиляции ранее.
Выделенные достоинства и недостатки обоих подходов являются существенными доводами в пользу предлагаемого подхода.
Результатом является ожидаемое устранение критических недостатков предшествующего подхода, связанных как с вопросом корректности генерируемого кросс-компилятором кода, так и долгосрочной поддержкой проекта и внесением в него изменений.
Важно отметить, что разрешение этих проблем в первую очередь достигается именно благодаря автоматизации всего процесса.
Также выделены выявленные недостатки созданного модуля, предложены пути их устранения.  
\par
Данная работа также была представлена в рамках 65-ой Всероссийской научной конференции МФТИ.
\newpage