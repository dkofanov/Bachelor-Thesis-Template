\chapter{Недостатки реализации}
\label{sec:Chapter4} \index{Chapter4}

После имплементации и интеграции описанного решения в основную ветвь проекта, проявились некоторые его тонкости, которые вполне можно отнести к недостаткам. В этой главе дан обзор этим моментам и предложены пути их сглаживания.

\section{Рассинхронизация флагов конфигурации}
Одна из существенных проблем, потенциально влекущая неконсистентность платформо-зависимых констант между кросс-компилятором и самой виртуальной машиной исходит из рассинхронизации флагов \textit{CMake}-конфигурации.
Для корректной конфигурации вспомогательных сборок важно передать все флаги, переопределённые во время инициирующего вызова \textit{cmake} из командной строки.
Если упустить какой-нибудь из них, то при конфигурации вспомогательной сборки он примет значение по-умолчанию, при этом такое различие в конфигурациях может повлиять, например, на наличие поля в какой-то структуре, смещения в которой отслеживаются рассматриваемым модулем, что повлечёт расхождение констант.
На момент написания работы, такая ошибка выявляется хотя и гарантировано, но достаточно поздно - в момент запуска приложения на устройстве.
Желательно более раннее детектирование, а в лучшем случае - предотвращение такой ошибки.
Поскольку \textit{cmake} не предоставляет интерфейса по определению флагов, указанных при вызове команды, решение может быть основанным на анализе файла \textit{CMakeCache.txt}, предназначение которого описывается в документации \textit{CMake}, содержащего действительные значения флагов. 

\section{Издержки по времени сборки}
При недостаточно аккуратно указанных зависимостях сборки проекта, возможна ситуация, при которой генерируемые файлы пересоздавались, но их содержание было бы тем же самым. Так как утилиты осуществляющие сборку (Make, Ninja) определяют модификацию файлов в терминах временных меток, а не их содержимого, это существенно замедляет инкрементальную сборку из-за лишней работы связанной с перекомпиляций, производящей в точности тот же самый результат.
Грубым решением этой проблемы будет использование утилиты на подобии \textit{ccache}.
Более правильным будет анализ зависимостей сборки и устранение ложных зависимостей.
Отсылаясь к опыту реализации и поддержки этого модуля, оказалось так, что эта проблема исходила из другой части проекта и проявлялась ранее, хотя и реже. Таким образом, анализ и исправление зависимостей привела к более предсказуемой инкрементальной сборке, устранив <<случайные>> перекомпиляции, занимающие время сравнимое с чистой сборкой проекта (т.е. около 10-20 минут).

\par
На данном этапе разработки, на конфигурацию одной вспомогательной сборки приходится около 10 секунд.
При сборке проекта с поддержкой нескольких платформ последовательная настройка может вносить некоторый дискомфорт.
Эту проблему можно решить с помощью параллелизации ценой потери информативности логов процесса конфигурации вспомогательных сборок, возможно даже совместно с их переносом из стадии конфигурации в стадию сборки основного проекта.
Это приводит к снижению наглядности процесса автогенерации и несколько усложняет трассировку проблем при ошибках конфигурации (например, связанных с инфраструктурой) для разработчика, не знакомого с этим модулем.
В этих условиях, предпочтение было сделано в пользу наглядности процесса.


\newpage